\section{引言}

\subsection{选题背景与意义}
\label{sec:background}
  安全传输层协议 (Transport Layer Security,下面简称 TLS ) 每天被全球数百万用户使用,作为互联网安全的核心构建块,不仅应用到浏览器的 HTTPS 协议中,还有其他应用层协议也使用了 TLS,比如 SSH、WSS 协议等。由于 TLS 1.2 及以下版本中的各种安全缺陷和设计缺陷\cite{7877537},\cite{8632026},会受到降级攻击、中间人攻击、BEAST 攻击、POODLE攻击等,最为严重的openssl心脏出血漏洞\cite{8001980},属于实现上的漏洞而不是协议上的漏洞,还有 TLS 1.2 的完整握手需要两轮往返时间,耗时长,即使使用会话恢复也需要一轮往返时间。
  
\subsection{国内外研究现状和相关工作}
\label{sec:related_work}
  有关 TLS 1.3 的讨论,多数停留在非正式发布前\cite{7883842}\cite{ARTICLE_typical1}\cite{ARTICLE_typical}的草案,基于 DH 密钥交换的零轮往返时长协议,但在正式发布前已经删除,最终使用的是基于 PSK 的握手恢复。在国内,只能在期刊上阅读到少量关于 TLS 1.3 的文章,而且文章内容讲述过于简单,不够全面,关于 TLS 1.3 协议最重要的部分零轮往返时长协议没有更多讲解,国内知名通信软件微信,参考 TLS 1.3 实现的安全通信协议 MMTLS\cite{MMTLS},其中实现零轮往返时长协议,比较于 TLS 协议,MMTLS 协议由于微信客户端每个人可用,于是删除了客户端认证相关的功能,同时在微信客户端程序中内置服务器的签名公钥,握手中不在需要进行服务器认证,减少发送的流量,关于 MMTLS 抗重放攻击,根据微信特有的后台架构,提出了基于客户端和服务器端时间序列的防重放策略,保证超过时间的重放包能被服务器拒绝,通过由 Proxy 层和 Logic 框架协同控制。
  
  支持 TLS 1.3的代码库,最广泛的当然有Openssl,Google 的 boringssl、guntls 等。国外相关工作,谷歌在基于 UDP 协议上实现的 QUIC Crypto 协议\cite{8280429}中首次实现了零轮往返时长协议,由于 QUIC 更早的使用零轮往返时长协议,实现的标准也提供给 TLS 1.3作为参考,但到了 TLS 1.3 正式版本发布,QUIC 反而会基于 TLS 1.3,并在以后的 HTTP/3.0 中使用,促进网络协议的发展。Facebook 使用 C++14 标准实现强大,高性能的 TLS 库,代码库命名为 Fizz,在 QUIC 基础改做出改动,在手机 APP 上实现零轮往返,实现更快的连接速度,并且有效地处理安全性问题,实现部署零轮往返时长协议,发现建立连接所需的时间降低了 41\%。
  
\subsection{研究内容及主要贡献}

分析 TLS 1.3 协议,利用 WebSocket 简单实现握手协议和记录层协议,重现 TLS 1.3 中 0-RTT 的重放攻击。实现对抗 0-RTT 重放的方法并测试效果。得到以下结果:

\begin{itemize}
  \item[-] TLS 1.3 比旧版本更安全可靠
  \item[-] 0-RTT 握手在连接时速度更快
  \item[-] 相关实现可以正确抵抗 0-RTT 重放攻击
\end{itemize}

\subsection{论文章节安排}
\label{sec:arrangement}

第一章引言

第二章实现简单 TLS 1.3 协议

第三章分析零轮往返时长协议中的重放攻击

第四章测试和实验结果

第五章总结与展望

这是一个与本文无关的table,只是为了方便后来人参考:
\begin{table}[H]
    \caption{\label{analysis:user_privilege:table}模块中用户的权限设置}
    \centering
    \begin{tabular}{*{5}{|l}|}
    \toprule
    模块&操作&管理员&课程教师&课程学生 \\ \hline
    \midrule
    \multirow{5}{*}{用户模块}&登入\/退出&是&是&是 \\
    &创建用户&是&否&否 \\
    &删除用户&是&否&否 \\
    &更新用户信息&是&是&是 \\
    &登陆后台&是&是&否 \\
    \midrule

    \multirow{4}{*}{课程模块}&创建课程&是&是&是 \\
    &删除课程&是&是&否 \\
    &修改课程信息&是&是&否 \\
    &查看、修改学生列表&是&是&否 \\
    \midrule

    \multirow{4}{*}{任务模块}&查看任务信息&是&是&是 \\
    &修改任务信息&是&是&否 \\
    &查看、修改提交&是&是&是 \\
    &提交评价&是&是&否 \\
    \midrule

    文件模块&上传、下载提交的文件&是&是&是 \\
    \midrule

    \multirow{3}{*}{镜像模块}&查看、修改镜像&是&是&否 \\
    &审核镜像&是&否&否 \\
    &启动镜像实例&是&是&是 \\
    \bottomrule
    \end{tabular}

\end{table}

\newpage